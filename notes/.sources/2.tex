\documentclass[12pt,notitlepage]{article}

\usepackage{hyperref}

\title{Project Euler: Problem 2
\\\small{\url{https://projecteuler.net/problem=1}}\\
\small{\url{https://github.com/oltdaniel/euler}}}

\begin{document}
  \maketitle

  \begin{center}
    \textit{"Each new term in the Fibonocci sequence is generated by adding the previous two terms. By starting with 1 and 2, the first 10 terms will be:\\1, 2, 3, 5, 8, 13, 21, 34, 55, 89, ...\\By considering the terms in the Fibonocci sequence whose values do not exceed four million, find the sum of the even-valued terms."}
  \end{center}

  The general idea to solve this, is starting an endless loop, that will break if the current fibonacii number exceeds
  $4,000,000$. During each iteration the numbers before will be added and stored, as well as moving the last Fibonocci
  number to the memory for the next iteration. THe current Fibonoccie will be then tested wether they are even or not
  and added to the sum if they are.\par

  As we already know, that only even fibonacii numbers will be added to the sum, this routine can be optimized by using
  some existing formulas.

  \subsection{Basic Idea}
  The basic idea is to start an endless loop, that calculates the next Fibonocci number, by adding the two previous
  ones. Followed by that, the new number, and the last one need to be stored for the next operation. After the current
  Fibonocci number has been proven to be even, it can be added to the final sum.\par
  \textbf{Complexity: $O(\log (\sqrt{5} * n) \div \log (\frac{1 + \sqrt{5}}{2}))$}

  \subsection{Memory Optimization}
  As the above solution requires the use of many memory operations, by moving the numbers around, this step
  is something that can be optimized with the \textit{Binet's Formula}. This allows us to calculate the $n$-th
  Fibonocci number without moving any values in the memory nor remembering any values calculated before. The formula
  is defined as follows:

  $$F_{n} = \frac{\varphi^{n} - \psi^{n}}{\sqrt{5}} = \frac{
    \frac{
      (1 + \sqrt{5})
    }
    {2}^{n} -
    \frac{
      (1 - \sqrt{5})
    }
    {2}^{n}
  }{
    \sqrt{5}
  }$$\par

  \textbf{Complexity: $O(\log (\sqrt{5} * n) \div \log (\frac{1 + \sqrt{5}}{2}))$}

  \subsection{Reduce Calculation}
  Based on the optimizations made above, we can now simplify the sequence of numers we will add in a loop, into
  a geometric series. This will allow us to calculate the sum with the following formula:

  $$S(n) = (F_{n + 2} - 1) \div 2$$

  However, as we only want even numbers we need to extend this. We know, that a Fibonocci number is only even, if the
  index of it is a multiple of $3$. We also know, that $n$ is the number of Fibonocci numbers under the set limit $4,000,000$.
  In order to calculate $n$ for $S(n \div 3)$ to find $s$, we can use the following formula based on \textit{Binet's Formula}:

  $$n = \log(\sqrt{5} * n) \div \log(\frac{1 + \sqrt{5}}{2})$$

  Followed by that, we can say, that the sum of all even Fibonocci numbers under $4,000,000$ is $s = S(n / 3)$.

  \textbf{Complexity: $O(1)$}
\end{document}
